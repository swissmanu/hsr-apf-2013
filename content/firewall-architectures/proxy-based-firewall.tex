\section{Proxy Based Firewall}
\label{sec:proxy-based-firewall}

Als Nachteil der ``\nameref{sec:packet-filter-firewall}'' wird erwähnt, dass lediglich der Inhalt des IP-Headers im Zuge der Überprüfung analysiert wird.

Die \emph{Proxy Based Firewall} fügt der ``\nameref{sec:packet-filter-firewall}'' spezifische Applikations-Proxies hinzu, welche den ein- und ausgehenden Traffic überprüfen und ggf. an den internen, eigentlichen Dienst weiterleiten.

Der interne Dienst wird auf diese Weise für den externen Host  komplett unsichtbar; er kommuniziert lediglich mit dem Proxy.


\subsection*{Kontext}
Netzwerkverkehr soll auf der Ebene des Application-Layers gefiltert werden können (vgl. ``\nameref{sec:packet-filter-firewall}'' tut dies lediglich auf dem Network-Layer). Auf diese Weise soll sichergestellt werden, dass keine schädlichen Befehl/schädlicher Code ins eigene Netz hinein gelangt resp. aus dem eigenen Netz heraus gesendet werden kann (Würmer, Trojaner etc.).

\subsection*{Problem}
Wie kann die ``\nameref{sec:packet-filter-firewall}'' so erweitert werden, dass nicht nur der IP-Header zur Filterung von Netzwerkverkehr verwendet werden kann? Wie kann auch der IP-Payload in die Filterung miteinbezogen werden?

Ergänzend zu den für die \nameref{sec:packet-filter-firewall} definierten Forces kommen folgende ergänzend hinzu:

\begin{itemize}
	\item In unserem Netzwerk werden verschiedenste Dienste angeboten. Entsprechend Umfangreich muss auch das Wissen der Firewall über die jeweiligen Dienste sein.
\end{itemize}


\subsection*{Lösung}
Die Firewall stellt für jeden zu schützenden Dienst einen Proxy zur Verfügung. Will ein fremder Host mit einem Dienst kommunizieren, kommuniziert er lediglich mit dem entsprechenden Proxy.

Aufgrund von definierten Regeln analysiert der Proxy den ein- oder ausgehenden Verkehr. Dabei bleibt es ihm frei überlassen ob er diesen weiterleiten, blockieren oder gar modifizieren will.

\subsection*{Vorteile}
\begin{itemize}
	\item Die \emph{Proxy Based Firewall} kann Netzwerkverkehr auf Applikationsebene filtern. Sie kann dabei gezielt auf applikationsspezifische Eigenheiten eingehen und die Kommunikation ggf. sogar verändern.
\end{itemize}

\subsection*{Nachteile}
\begin{itemize}
	\item Für jeden Dienst wird eine konkrete Proxyimplementierung benötigt.
	\item Das Betreiben der Proxies sowie die genauere Analyse des kompletten IP-Pakets führt zu höheren Kosten sowie tendenziell höherem Performance Overhead.
	\item Erhöhte Komplexität aufgrund der zusätzlichen Sicherungsebene.
\end{itemize}


\subsection*{Reallife Beispiele}
\begin{itemize}
	\item NAT - Network Address Translation
\end{itemize}

\subsection*{Mögliche Prüfungsfragen}
\begin{itemize}
	\item \emph{Nennen Sie konkrete Anwendungen für die Proxy Based Firewall.}\\
	Zugang zu bestimmten Internetseiten blockieren (HTTP Proxy), ``Network Address Translation'' um die interne Netzwerkstruktur zu verschleiern. Idee: Telnet-Proxy welcher gewisse Kommandos nicht zulässt.
\end{itemize}