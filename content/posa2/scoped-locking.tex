\section{Scoped Locking}


The \emph{Scoped Locking} C++ idiom ensures that a lock is acquired when control enters a scope and released automatically when control  leaves the scope, regardless of the return path from the scope

\subsection*{Kontext}

Shared resources welche gleichzeitig von mehreren Threads manipuliert werden

\subsection*{Problem}

Um eine Methode Thread-Safe zu machen wird vielfach z.B. ein Mutex innerhalb der Methode acquired. Da eine Methode aber mehrere Return-Paths haben kann (u.a. auch Exceptions, continue, break,  etc.), wird dies pro Zeile Code immer wie schwieriger.

\subsection*{Lösung}

Erstelle eine Klasse, welche im Konstruktur ein Lock acquired und im Destruktor dieses wieder freigibt. Diese Klasse dann jeweils in Methoden/Block scopes von Kritischen Code-Teilen instantieren.
Der Destruktor wird automatisch aufgerufen wenn die geschützte Methode verlassen wird. Somit ist sichergestellt, dass Locks immer wieder freigegeben werden.

\subsection*{Varianten}

\begin{itemize}
	\item \emph{Explicit Accessors}: Ein Nachteil dieser Implementation: Ein Lock wird immer nur nach dem verlassen der geschützten Methode freigegeben. Falls es gewünscht ist, explizit dieses freizugeben, müsste eine public Methode \emph{release()} in der Guard Klasse erstellt werden. Über diese Methode kann dann explizit das Lock freigegeben werden.
\end{itemize}

\subsection*{Known Uses}

\begin{itemize}
	\item Java: \emph{synchronized} Methoden.
\end{itemize}

\subsection*{Vorteile}

\begin{itemize}
	\item Increased Robustness
\end{itemize}

\subsection*{Nachteile}

\begin{itemize}
	\item Potential für Deadlocks falls es rekursiv verwendet wird
	\item Limitationen aufgrund von Sprach-spezifischer Semantik (whenn das \emph{C}-Feature \emph{longjmp} verwendet wird, werden keine Destruktoren aufgerufen)
\end{itemize}

