\section{I\&A Requirements}

Muss ein I\&A Service etabliert werden, hilft das I\&A Requirements Pattern mit seinen generischen Requirementsvorlagen bei der Analyse eines bestehenden oder zu konzipierenden Systems.

Dabei werden nicht nur sicherheitsrelevante Faktoren berücksichtigt. Aspekte wie Kosteneffektivität oder Benutzerzufriedenheit und -akzeptanz fliessen ebenso in die Analyse mitein.

\subsection*{Kontext}
Eine Organisation oder ein Projekt konzipiert die Verwendung von I\&A. Das Pattern unterstützt die Analyse jeglicher Situationen, in welchen sowohl Identification als auch Authorization notwendig ist.

\subsection*{Problem}


\subsection*{Lösung}


\subsection*{Erweiterungen}

\subsection*{Vorteile}
\begin{itemize}
	\item 
\end{itemize}

\subsection*{Nachteile}
\begin{itemize}
	\item 
\end{itemize}

\subsection*{Beispielanwendungen}
\begin{itemize}
	\item 
\end{itemize}

\subsection*{Mögliche Prüfungsfragen}
\begin{itemize}
	\item \emph{adasd?}\\
	dasd
\end{itemize}