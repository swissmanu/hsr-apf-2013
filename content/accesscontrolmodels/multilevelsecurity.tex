\section{Multilevel Security}

Oft sollen Informationen in verschiedene Sicherheitskategorien einsortiert werden: Ein Unternehmen möchte bspw. nicht, dass der neue Praktikant auf strategisch wichtige Informationen aus dem Verwaltungsrat-Meeting zugreifen kann. Das \emph{Multi Level Security} Pattern beschreibt wie Informationen klassifiziert werden können.

Es definiert hierzu \emph{Policies} welche Subjekten \emph{Clearances} für bestimmte \emph{Sensitivity Levels} erteilt.


\subsection*{Kontext}
Sicherheitskritische Informationen resp. deren Verwahrung erfordert erhöhten Aufwand im Sicherheitskonzept.

\subsection*{Problem}
Es gibt es unterschiedlich sensitive Informationen. Ein Subjekt soll entsprechend seiner Stellung innerhalb der Organistaionsstruktur Zugriff auf kritische oder weniger kritische Informationen Zugriff erhalten.

Dabei soll ein Maximum an Flexibilität für das Verändern von Parametern bestehen:
\begin{itemize}
	\item Ein Subjekt soll so einfach wie möglich einer anderen Stufe in der Organisation zugewiesen werden könne
	\item Die Sensitivität einer Information muss so einfach wie möglich angepasst werden können
\end{itemize}

\subsection*{Lösung}
Jeder Information wird ein \emph{Sensitivity Level} zugwiesen. \emph{Policies} definieren, welche Elemente aus der Organistaionstruktur auf welche \emph{Sensitivity Levels} zugriff erhalten.

\emph{Policies} werden von \emph{Trusted Processes} erstellt und verwaltet. 



\begin{figure}[H]
	\begin{center}
	\begin{tikzpicture}
		\umlemptyclass[x=0,y=0]{TrustedProcess}
		
		\umlemptyclass[x=-3,y=-3]{Subject}
		\umlemptyclass[name=SubjectCategory,x=-5,y=-6]{Category}
		\umlemptyclass[x=-2,y=-6]{ClearanceLevel}

		\umlemptyclass[x=3,y=-3]{Data}
		\umlemptyclass[name=DataCategory,x=6,y=-6]{Category}
		\umlemptyclass[x=2.7,y=-6]{ClassificationLevel}


		\umlassoc[arg2=AssignLevel,geometry=-|,mult1=*,mult2=*,pos1=0.3,pos2=1.9]{TrustedProcess}{Subject}
		\umlassoc[arg2=AssignLevel,geometry=-|,mult1=*,mult2=*,pos1=0.3,pos2=1.9]{TrustedProcess}{Data}

		\umlassoc[arg=canAccess,pos1=0.1,pos2=1,mult1=*,mult2=*,align2=right]{Subject}{Data}

		\umlcompo[geometry=|-|,mult=*,pos=2.8]{Subject}{SubjectCategory}
		\umlcompo[geometry=|-|,mult=1,pos=2.7]{Subject}{ClearanceLevel}

		\umlcompo[geometry=|-|,mult=*,pos=2.8]{Data}{DataCategory}
		\umlcompo[geometry=|-|,mult=1,pos=2.7]{Data}{ClassificationLevel}
	\end{tikzpicture}
	\end{center}
\caption{Multilevel Security}
\end{figure}

\subsection*{Vorteile}
\begin{itemize}
	\item Welcher Benutzer welche Berechtigung erhalten soll kann relativ einfach am Organigramm einer Organisation abgeleitet werden.
	\item Durch die Modellierung der \emph{Trusted Processes} trennt dieses Pattern strikt zwischen Administration und tatsächliche Umsetzung Sicherheitsregeln.
\end{itemize}

\subsection*{Nachteile}
\begin{itemize}
	\item Bei der Umsetzung dieses Patterns sollte darauf geachtet werden, dass normierte Bezeichnungen für die entsprechenden Sensitivity und Clearance Levels verwendet wird (--> Glossar)
	\item Der definierte Trusted Process muss auch als solcher umgesetzt werden (Prozessdefinition als auch effektive Umsetzung)
	\item Daten als auch Benutzer müssen optimalerweise in hierarchische Berechtigungstrukturen eingeteilt werden können.
	Insbesondere in kommerziellen Umgebungen ist dies schwierig bis fast unmöglich (vgl. Militär)
\end{itemize}

\subsection*{Erweiterungen}
Das Rollenkonzept von \ref{sec:rbac} Role Based Access Control kann mit diesem Pattern problemlos kompiniert werden: Dabei werden die \emph{Clearance Levels} einfach auf die Gruppen statt direkt auf die Benutzer zugwiesen.

\subsection*{Beispielanwendungen}
\begin{itemize}
	\item Militäreisches IT-System
	\item Datenbanksysteme (bspw. Oracle)
	\item Betriebssysteme (bspw. HP Virtual Vault: HP Unix Abkömmling, properitär)
\end{itemize}