\subsection{Role Rights Definition}
Beim Definieren von Sicherheitsrichtlinien spielt das \emph{Least Privilege} oder auch das \emph{Need to know} Prinzip eine fundamentale Rolle: Jedes Subjekt soll gerade so viele Berechtigungen erhalten, damit es seine Aufgaben ungehindert erledigen kann.

Das \emph{Role Rights Definition} Pattern beschreibt einen systematischen Ansatz, wie aus vorhandenen \emph{Requirements Engineering} Artefakten \emph{Need to Know}-konforme Sicherheitsregeln gewonnen werden können

\subsection*{Kontext}
Eine relativ komplexe Ansammlung von Rollen soll mit passenden Berechtigungen ausgestattet werden.

\subsection*{Problem}
\emph{Role Based Access Control} wird in vielen Systemen als grundlegendes Sicherheitkonzept verwendet. Wie im Abschnitt \ref{sec:rbac} erwähnt ist die Definition von Berechtigungskonzepten bei umfangreichen System (und grosser Anzahl an Aufgabenbereichen) mit beträchtlichem Aufwand verbunden.

Zudem überlässt \emph{Role Based Access Control} es komplett dem Implementator, aufgrund von welchen Informationen Gruppen resp. deren Berechtigungen zusammengestellt werden.


Wie können wir \emph{Role Based Access Control} mit Sicherheitsrichtlinien füttern, welche folgende Punkte befriedigen?
\begin{itemize}
	\item Rollen sollen Aufgabenbereichen in der Organisationsstruktur entsprechen
	\item Rechte sollen so erteilt werden, dass sie dem \emph{Need to know} Prinzip genügen
	\item Weiterhin soll die Anpassung bestehender Rollen und Rechten so einfach wie möglich bleiben
	\item Die Definition von Rechten und Rollen soll unabhängig von einer effektiven Implementierung des Systems bleiben
\end{itemize}


\subsection*{Lösung}
Die Idee ist denkbar einfach: Ein (hoffentlich bestehendes) Use Case Model und die damit verbundenen Sequenzdiagramme werden dazu verwendet, alle von \emph{Role Based Access Controls} benötigten Elemente zu erfassen:

\begin{itemize}
	\item Ein \emph{Actor} entspricht einer \emph{Role}
	\item Jegliche \emph{Objects} entsprechen einem potentiellen \emph{ProtectionObject}
	\item Jede \emph{Operation} welche ein \emph{Actor} auf einem \emph{Object} ausführt, ist ein potentielles \emph{Right} einer \emph{Role}
	\item Eine \emph{Use Case Exception} bestimmt das Verhalten im Falle einer Verletzung einer Sicherheitsrichtlinie
\end{itemize}


\subsection*{Vorteile}
\begin{itemize}
	\item Sicherheitsrichtlinien können, bei entsprechendem Projektvorgehen, bereits sehr früh definiert und erkannt werden.
	\item Wird ein ``\emph{model driven}''-Ansatz für die Softwareentwicklung gewählt, können Sicherheitsrichtlinien im optimalsten fall ``einfach'' aus den bestehenden Requirements Artefakten generiert werden
	\item \emph{Role Rights Definition} erstellt ``perfekte'' Sicherheitsrichtlinien für \emph{\gls{RBAC}}
	\item Sind alle Use Cases modelliert, und das System kann auf diese Weise komplett abgebildet werden, so ist ein Maximum an Sicherheit garantiert
	\item Verändert sich die Funktionalität (sprich die Use Cases) des Systems (neuer Release etc.), so können auch die damit verbundenen Änderungen im Sicherheitskonzept problemlos abgebildet werden.
	\item \emph{Role Rights Definition} bleibt komplett implementationsneutral
\end{itemize}

\subsection*{Nachteile}
\begin{itemize}
	\item Ohne ausführliches, durchgehendes und kompetentes Requirements Engineering hat dieses Pattern so gut wie keinen Nutzen
\end{itemize}

\subsection*{Mögliche Prüfungsfragen}
\begin{itemize}
	\item \emph{Für welches Pattern ist der ``Output'' von Role Rights Definition bestens geeignet? Warum?}\\
	\emph{Role Rights Definition} analysiert Use Cases und extrahiert daraus aufgaben- und funktionsbezogene Zugriffsberechtigungen für alle vorhandenen \emph{Actors}.\\
	Diese Regeln entsprechen dem \emph{Need to know} Prinzip: Jeder \emph{Actor} kann genau das tun/sehen, was er zu Ausübung seiner Aufgaben tun/sehen können muss.\\Damit sind eben diese Regeln optimal für die Verwendung im \emph{\gls{RBAC}} Pattern geeignet.

	\item \emph{Warum reicht es nicht aus, lediglich das Use Case Model zur Gewinnung von Roles und Rights zu analysieren?}\\
	Die Sequenzdiagramme geben detailierte Auskunft darüber, zu welchem Zeitpunkt welcher \emph{Actor} welches \emph{Right} für welches explizite \emph{Protection Object} benötigt. Ohne diese Informationen ergibt sich ein unvollständiges Gesamtbild.
\end{itemize}