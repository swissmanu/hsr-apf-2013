\subsection{Controlled Object Factory}
\label{sec:controlled-object-factory}

Die \emph{Controlled Object Factory} stellt sicher, dass neu erstellten Objekten (Files etc.) automatisch korrekte Rechte zugeteilt werden.

\subsection*{Kontext}
Ein System muss den Zugriff auf erstellte Objekte kontrollieren können. Dabei kommen Zugriffsregeln zum Zug, welche von der Klassifizierung der entsprechenden Objekte abhängig ist.

\subsection*{Problem}
Prozesse erstellen neue Objekte wie bspw. Dateien. Die Zugriffsregeln dieser Objekte sollen bei der Erstellung dieser gesetzt werden um unberechtigten Zugriff von Beginn an zu verhindern.

Ein weiterer Faktor stellen gepoolte Objekte dar: Zum Zeitpunkt der Zuweisung zu einem konkreten Prozess soll diesem auch gleich dynamisch der Zugriff gewährt werden.

\begin{itemize}
	\item Prozesse erstellen verschiedene Arten von Objekten. Die Zuweisung von Zugriffsrechten soll jedoch generisch behandelt werden können.
	\item Für gepoolte Objekte soll eine dynamische Zuweisung von Zugriffsrechten ermöglicht werden
	\item Es gibt Richtlinien welche vorgeben, welche Zugriffsrechte neue Objekte erhalten sollen
\end{itemize}


\subsection*{Lösung}
Jedem Objekt welches neu erstellt wird, wird automatisch und zentralisiert ein Set von Zugriffsrechten zugewiesen.


\subsection*{Vorteile}
\begin{itemize}
	\item Es gibt keine neuen Objekte mehr, welche Standardzugriffrechte zugewiesen haben, weil jemand vergessen hat, diese zu überschreiben.
	\item Es können Richtlinien definiert werden, welche Objekte wie geschützt werden sollen
	\item Ergänzend kann ein Betriebssystem eine \emph{Ownership Policy} einführen. Der Besitzer eines Objekts hat dann bspw. alle möglichen Berechtigungen an einem Objekt.
\end{itemize}

\subsection*{Nachteile}
\begin{itemize}
	\item Es entsteht ein entsprechender Overhead für die Definition der Rechte
\end{itemize}