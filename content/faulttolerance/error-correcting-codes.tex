\section{Error Correcting Codes}

\subsection{Ausgangslage}

Wie können Daten möglichst Fehlerfrei gehalten werden bzw. wie können fehlerhafte Daten möglichst schnell korrigiert werden?

\subsection{Lösungsansatz}

Mit dem Checksum Ansatz kann das System relativ schnell erkennen ob Daten korrekt sind, jedoch nicht welcher Teil ungültig ist. Deshalb müssen zusätzlich Korrekturbits eingeführt werden, womit das System erkennen kann welcher Teil zerstört bzw. ungültig ist.

\subsection{Schlussfolgerung}

Speichere mit jeder Checksum möglichst viele Informationen, damit fehlerhafte Daten möglichst korrigiert werden können.

